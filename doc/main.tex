\documentclass{article}
\usepackage[utf8]{inputenc}

\usepackage{graphicx}

\usepackage{makecell}
\usepackage{amssymb}
\usepackage{amsmath}
\usepackage{stmaryrd}
\usepackage{geometry}

\geometry{
  textwidth=138mm,
  textheight=215mm,
  left=20mm,
  right=20mm,
  top=25.4mm, 
  bottom=25.4mm,
  headheight=2.17cm,
  headsep=4mm,
  footskip=12mm,
  heightrounded,
}


\newtheorem{theorem}{Theorem}[section]
\newtheorem{corollary}[theorem]{Corollary}
\newtheorem{dfn}{Definition}
\newtheorem{hypothesis}{Hypothesis}
\newtheorem{lemma}[theorem]{Lemma}
\newtheorem{proposition}[theorem]{Proposition}
\newtheorem{remark}{Remark}
\newtheorem{example}{Example}
\newtheorem{conjecture}{Conjecture}
\newenvironment{proof}{\textbf{PROOF}}{\hfill $\square$ \vspace{1em}}


\usepackage{xcolor}
\usepackage{braket}


\title{\textbf{Rules of QPV2}}
\author{Yingte Xu}
\date{2023 October}

\begin{document}

\maketitle

Types in the prover include the following cases.
\begin{itemize}
    \item The types for operators are tuples: $(qnum, property...)$.
    \item The types for indexed operators are tuples: $(qvar, property...)$.
    \item The types for programs and proof are denoted as $PType$.
\end{itemize}
The property used here include Hermitian property, projective property, unitary property, semi-positive property and partial density operator property.

\vspace{1em}

Terms in the prover include the following cases.
\begin{itemize}
    \item Constant, hereafter denoted by $C$, are terms.
    \item Operators, hereafter denoted by $O$, are terms.
    \item Indexed operators (labelled operators), hereafter denoted by IO, are terms.
    \item Programs and proofs, hereafter denoted by $P$, are terms.
\end{itemize}

And $c$ denotes real numbers.


$$
\begin{aligned}
    P ::=\ & \mathbf{abort}\ |\ \mathbf{skip}\ |\ qvar :=0\ |\ IO\ \\
        & |\ \mathbf{assert}\ IO\ \\
        & |\ [\mathrm{pre} : IO, \mathrm{post} : IO] \\
        & |\ P; P \\
        & |\ ( P\ \_\ c \oplus P ) \\
        & |\ \mathbf{if}\ IO\ \mathbf{then}\ P\ \mathbf{else}\ P\ \mathbf{end}\\
        & |\ \mathbf{while}\ IO\ \mathbf{do}\ P\ \mathbf{end}\\
        & |\ \mathrm{Prog}\ C\\
        & |\ [\mathrm{pre} : IO, \mathrm{post} : IO] ==> P.
\end{aligned}
$$

$$
\begin{aligned}
    O ::=\ & m\ |\ C\ \\
        & |\ - O\ |\ O + O\ |\ O - O\\
        & |\ c * O\ |\ O * O\ |\ O^\dagger \\
        & |\ O \otimes O\ \\
        & |\ O \vee O\ |\ O \wedge O\ |\ O ^\bot \\
        & |\ O \rightsquigarrow O\ |\ O \Cap O.
\end{aligned}
$$

Here $m$ denote a Python matrix instance.

$$
\begin{aligned}
    IO ::=\ &\textbf{IQOpt}\ C\ |\ O\ qvar \\
        & |\ - IO\ |\ IO + IO\ |\ IO - IO\\
        & |\ c * IO\ |\ IO * IO\ |\ IO^\dagger \\
        & |\ IO \otimes IO\ \\
        & |\ IO \vee IO\ |\ IO \wedge IO\ |\ IO ^\bot \\
        & |\ IO \rightsquigarrow IO\ |\ IO \Cap IO.
\end{aligned}
$$

\textbf{Definition}: A definition is a triple of a constant, a term and its type. It is written as $C := t : T$.

\textbf{Environment}: An environment is a ordered list of definitions. We denote a well-formed environment as $\mathcal{WF}(E)$.

$$
\frac{}{\mathcal{WF}([])}
$$

$$
\frac{E \vdash t : T\quad C \notin E}{\mathcal{WF}(E; C := t : T)}
$$



In principle we need a type system for the Dirac notations (Hilbert space, Hermitian property, projective property...).
Type checking and inferrence of operators is implemented, although not that formal. Type checking include the check of quantum variables and corresponding properties.
For example,

$$
\frac{\mathcal{WF(E)}}{E \vdash \mathbf{skip} : PType}
$$

$$
\frac{\mathcal{WF}(E)}{E \vdash m : (qnum, ...)}
$$

$$
\frac{\mathcal{WF}(E)\quad C:=t:T \in E}{E \vdash C : T}
$$

$$
\frac{\mathcal{WF}(E)\quad E \vdash O_1 : (qnum, \text{unitary}) \quad E \vdash O_2 : (qnum, \text{unitary})}{E \vdash O_1 * O_2 : (qnum, \text{unitary})}
$$

$$
\frac{\mathcal{WF}(E)\quad E \vdash IO_1 : (Q_1, \text{projective}) \quad E \vdash IO_2 : (Q_2, \text{projective})}{E \vdash IO_1 \wedge IO_2 : (Q_1 \cup Q_2, \text{projective})}
$$

$$
\cdots
$$

\textbf{Evaluation}: An environment can evaluate an operator term to the Python matrix instance. This is defined by:

$$
\frac{\mathcal{WF}(E)\quad C:=m \in E}{E(C) = m}
$$

$$
\frac{\mathcal{WF}(E)\quad E(O_1) = m_1\quad E(O_2) = m_2 }{E(O_1 * O_2) = m_1 * m_2}
$$

$$
\cdots
$$

\textbf{Construct the refinement proof}

$$
\frac{\mathcal{WF}(E)\quad E(IO_1) \sqsubseteq wlp.P.E(IO_2)}{E \vdash [\mathrm{pre} : IO_1, \mathrm{post} : IO_2] ==> P : PType}
$$

Corollaries:

$$
\frac{E \vdash [\mathrm{pre} : IO_1, \mathrm{post} : IO_2]:PType \quad E \vdash IO_3 : (Q, projective)}{E \vdash [\mathrm{pre} : IO_1, \mathrm{post} : IO_3]; [\mathrm{pre} : IO_3, \mathrm{post} : IO_2] : PType}
$$

$$
\frac{E \vdash Inv : (Q_1, projective) \quad E \vdash P : (Q_2, projective)}{E \vdash [\mathrm{pre} : Inv, \mathrm{post} : P^\bot \Cap Inv]==> \mathbf{while}\ P\ \mathbf{do}\ [\mathrm{pre} : P \Cap Inv, \mathrm{post} : Inv]\ \mathbf{end} : PType }
$$



\end{document}
